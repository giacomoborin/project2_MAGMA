\documentclass{article}
\usepackage[T1]{fontenc}
\usepackage[utf8]{inputenc}
\usepackage[english]{babel}
\usepackage{amssymb,amsmath,amsthm,mathtools} 
\usepackage{tikz-cd,wrapfig}
\usepackage{tcolorbox}
\usepackage{graphicx}
\usepackage{pdfpages}
\usepackage{enumitem}
\usepackage{pdfsync}
\usepackage{amstext} % for \text macro
\usepackage{array}   % for \newcolumntype macro
% \newcolumntype{C}{>{$}c<{$}} % math-mode version of "C" column type
%Personal package
\usepackage{listings}
\usepackage{lstlang0}
\usepackage{bm}
\usepackage{csquotes}


%Commands
\newcommand{\DD}{\mathcal{D}}
\DeclareMathOperator{\sgn}{sgn}
\newcommand{\WW}{\mathcal{W}}
\newcommand{\FT}{\mathcal{F}}
\DeclareMathOperator{\ifff}{\iff}


% Theorem definitons 
\theoremstyle{plain}
\newtheorem{teo}{Theorem}[section]
\newtheorem{lem}[teo]{Lemma}
\newtheorem{prop}[teo]{Proposition}
\newtheorem{cor}[teo]{Corollary}
\newtheorem*{form}{Formula}

\theoremstyle{remark}
\newtheorem{rem}{Remark}
\newtheorem{rems}[rem]{Remarks}

\theoremstyle{definition}
\newtheorem{deff}[teo]{Definiton}
\newtheorem{idea}{Idea}
\newtheorem*{nota}{Notation}

%Bibliography
\usepackage[backend=biber, style=alphabetic, maxnames = 8, giveninits=true,isbn=false,eprint=false]{biblatex}
%other_styles:numeric_authortitle
\addbibresource{biblio.bib}


\usepackage{hyperref}



\title{Report for the MAGMA project}
\author{Giacomo Borin}

\begin{document}

\maketitle

\begin{abstract}
	The report contain the implementation of 5 algorithms:
	\begin{itemize}
		\item ECDSA, a digital signature scheme based on algebraic curves.
		\item Pohlig Hellman alorithm for solving discrete logarithm problem for Algebraic curves. Particularly efficient when the base point has order with only small primes in the factorization
		\item Index Calculus algorithm for solving discrete logarithm problem for in finite fields of prime order using linear algebra and B-smooth sieving.
		\item Solovay-Strassen primality test, based on the evaluation of the Jacobi Symbol.
		\item Lehman factorization algorithm, based on a modification of the Fermat factorization algorithm.
	\end{itemize}

\end{abstract}


\section{ECDSA}

We have implemented the algorithm from~\cite[Section 6.6]{washington}.

\lstinputlisting[language=magma, firstline=17, lastline=38]{../ECDSA.mag}
\lstinputlisting[language=magma, firstline=41, lastline=60]{../ECDSA.mag}

\subsection{Implementation choiches}

- following the washington

It would be faster to use $k = 4$ as Sony, since it simplifies a lot the evaluation of $kG$ (in fact it would require only two doublings) but it would be very bad for security. However I have tried to use again the idea at the commit \href{https://github.com/giacomoborin/project2_MAGMA/commit/1209afbbb3bed86d26ebf342bf6f43453de3e795}{1209afbbb3bed86d26ebf342bf6f43453de3e795},

Some evaluation can be seen in the \href{https://github.com/giacomoborin/project2_MAGMA/pull/36}{associated pull request}.

\begin{lstlisting}[language = magma]
> load "ECDSA.mag"; // k random
Loading "ECDSA.mag"
> load "Test_ECDSA.mag";
Loading "Test_ECDSA.mag"
Time: 6.130
> load "ECDSA.mag"; // k = 4    
Loading "ECDSA.mag"
> load "Test_ECDSA.mag";
Loading "Test_ECDSA.mag"
Time: 4.230
\end{lstlisting}


Another simple idea is to evaluate only once the inverse of $s$, it saves only $0.1$ seconds, but it is still worth it (you can see more on \href{https://github.com/giacomoborin/project2_MAGMA/issues/13}{the relative issue on github}).


\newpage
\printbibliography


\end{document}
